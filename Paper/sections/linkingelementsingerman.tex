\section{Linking elements in German}
\label{sec:linkingelementsingerman}

\subsection{The form and distribution of linking elements in German}
\label{sec:theformanddistributionoflinkingelementsingerman}

In this paper, we exclusively deal with determinative (endocentric) nominal compounds, \ie compounds formed through the concatenation of two constituent nouns where the second noun (\textit{N2}) is the head modified by the first noun (\textit{N1}).
While compounding is a recursive process in principle, we ignore complications involved with more complex structures and look only at binary compounds.
We refer to these structures as \textit{N1+N2 compounds}.


We use the notation \textit{-X} to denote linking element \textit{X} attaching to N1 and \textit{=X} to denote a linking element attaching to N1 with additional umlaut.
Furthermore, \textit{+} separates N1 (possibly with linking element) and N2 in our analyses.
The special notation \textit{*e} is used for a linkage where the final schwa of N1 is deleted.
Orthographically, German compounds are usually spelled as one word.
Table~\ref{tab:linkages} shows the different types of linkages in German.

\begin{table}[htbp]
  \resizebox{\textwidth}{!}{
    \begin{tabular}{llll}
    \toprule
    Linkage & example & literal gloss & translation \\
    \midrule
    ∅    & \textit{Haus-tür}                & house door       & `front door' \\
    -s   & \textit{Anfang-s+zeit}           & beginning time   & `initial period' \\
    -n   & \textit{Katze-n+pfote}           & cat paw          & `cat's paw' \\
    -en  & \textit{Frau-en+stimme}          & woman voice      & `female voice' \\
    *e   & \textit{Kirsch+kuchen}           & cherry cake      & `cherry cake' \\
    -e   & \textit{Geschenk-e+laden}        & gift store       & `gift store' \\
    =e   & \textit{Händ=e+druck}            & hand press       & `handshake' \\
    =    & \textit{Mütter=+zentrum}          & mother centre    & `centre for mothers' \\
    -er  & \textit{Kind-er+buch}            & child book       & `children's book' \\
    =er  & \textit{Büch=er+regal}           & book shelf       & `bookshelf' \\
    -ns  & \textit{Name-ns+schutz}          & name protection  & `name protection' \\
    -ens & \textit{Herz-ens+angelegenheit}  & heart matter     & `affair of the heart' \\
    \bottomrule
    \end{tabular}
  }
  \caption{Overview of the different linkages in German nominal compounds; \textit{=} is used to denote linking elements which trigger umlaut on N1, \textit{*e} denotes linkages where the final schwa of N1 is deleted}
  \label{tab:linkages}
\end{table}

Whether \textit{-en} and \textit{-n} and \textit{-ens} and \textit{-ns} are allomorphic variants is debated.
See Sections~3.2 and~3.6 of \textcite{NueblingSzczepaniak2013} for an argument in favour of this view and \textcite[33--36]{Neef2015} for an argument against it.
This issue does not affect our study, and we separate the two potential alloforms in order to maximise the informativity of our results.

\subsection{Do linking elements have a function, semantics, or neither?}
\label{sub:dolinkingelementshaveafunctionsemanticsorneither}


\paragraph{Final statement, copied from elsewhere}

Therefore, the main hypothesis which we investigate is whether a pluralic link in a N1+N2 compound can be interpreted as a semantic plural by native speakers of German.
Since annotating corpus exemplars for whether the N1 in a compound has a plural meaning is quite difficult in the general case, we isolate two specific and easy to detect configurations in which a pluralic linking element might signal semantic plurality:
(i) a plural on the entire compound might trigger a plural interpretation of N1 and subsequently the use of a PL (external plural);
(ii) certain N2s combined with an appropriate semantic relation between N1 and N2 might force N1 to have a plural interpretation and therefore lead to a preference for using the PL (internal plural).
In Section~\ref{sec:corpusstudy}, we study these two more specific hypotheses using corpus data.
Then, in Section~\ref{sec:split100experiment}, we examine them in an experimental paradigm (split-100 ratings).
Before we turn to these studies, Section~\ref{sec:data} describes the exploratory work that allowed us to make a reasonable selection of items for the studies.
