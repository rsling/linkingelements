\section{Linking elements in German}
\label{sec:linkingelementsingerman}

\subsection{The form and distribution of linking elements in German}
\label{sec:theformanddistributionoflinkingelementsingerman}

\subsubsection{German linking elements}

In this paper, we exclusively deal with determinative (endocentric) nominal compounds, \ie compounds formed through the concatenation of two constituent nouns (possibly with intervening linking elements) where the second noun (N2) is the head modified by the first noun (N1).
While compounding is a recursive process in principle, we ignore complications involved with more complex structures and look only at binary compounds.
We refer to these structures as \textit{N1+N2 compounds}.

In the majority of such compounds in German, N1 and N2 are simply concatenated, and N1 appears in its base (uninflected) form, which is identical to at least the nominative singular form.
A linking element, as mentioned above, is (in its typical form) segmental material inserted between N1 and N2.
Linking elements in compounds are not exclusive to German; see \textcite{SchreuderEa1998,BangaEa2013a,BangaEa2013b} on Dutch linking elements and see also \textcite[27]{KrottEa2007} for a typological overview of some systems of compound infixoids.
However, German speakers use an unusually high number of different linking elements, and they may also make other types of changes to the base form of N1.
The list of segmental linking elements is widely accepted to be: \textit{e}, \textit{er}, \textit{s}, \textit{es}, \textit{n}, \textit{en}, \textit{ns}, and \textit{ens} (see \citealt[31]{Neef2015}, \citealt{KrottEa2007}).
The elements \textit{e} and \textit{er} can occur together with an optional umlaut on N1, and some compounds only have umlaut on N1 and no segmental linking element at all.
Also, certain linking elements can replace N1-final segments.
Lastly, N1s ending in schwa (graphemically \textit{<e>}) can drop this schwa in compounds.
Even though these are, strictly speaking, not all cases of additional material being inserted into the compound, we will maintain the conventions from the literature and use the term ``linking element'' as an umbrella term to refer to any formal changes to N1 which occur in the compounding process.
Where the distinction is relevant, we will use more specific terminology (\ie segmental linking elements, umlaut, replacement, deletion).

Before we continue, a few notes on the notation we use in this paper:
We will not mark zero linking elements as, for example, $\varnothing$, but rather represent the simply concatenated compounds as N1+N2 without anything in between.
We will represent a segmental linking element \textit{X} attaching to N1 as \textit{-X}, and \textit{=X} will denote a segmental linking element attaching to N1 with additional umlaut in N1.
An umlaut-only linking element is indicated by \textit{=} alone.
The special notation \textit{*e} is used for deletion of the final schwa of N1.
Furthermore, \textit{+} separates N1 (possibly with linking element) and N2 in our analyses.
Graphemically, however, German compounds are usually spelled as one word; the segmentation is intended purely as a reading aid.%
	\footnote{Some compounds, for example those involving names or loan words, are sometimes written with a hyphen, such as \textit{Adenauer-Zeit} `Adenauer period'.
	We exclude them from our study because they virtually never occur with linking elements and have specific compound-internal semantics.
	There are also occasional spellings of compounds as two words such as \textit{Blumen Welt} `world of flowers' and with in-word capital letters such as \textit{BlumenWelt}, which are much debated from a normative perspective (see also \citealt{Scherer2012}).
	These spellings are assumed to be specific to certain genres (such as adverts) and are quite rare overall.
	Therefore, we do not include them in our study.}
Table~\ref{tab:linkages} illustrates the different types of linking elements in German in N+N compounds.%
	\footnote{Linking elements are also found in other sorts of German compounds, for example N+A compounds such as \textit{herzensgut} (\textit{Herz-ens+gut}) `kindhearted', \textit{lebensfreundlich} (\textit{Leben-s+freundlich}) `life-sustaining\slash livable', or \textit{hundelieb} (\textit{Hund-e+lieb}) `dog-loving'.
	While their distribution, function, and interpretation in such compounds might be related to that in N1+N2 compounds, our studies reported in Sections~\ref{sec:corpusstudy} and~\ref{sec:split100experiment} make use of diagnostic features exclusive to N1+N2 compounds, and we therefore excluded all other types of compounds.}

\begin{table}[h!  ]
  \resizebox{\textwidth}{!}{
    \begin{tabular}{lllll}
    \toprule
    Element & $\equiv$P & example & literal gloss & translation \\
    \midrule
    $\varnothing$    && \textit{Haus+tür}                & house door       & `front door' \\
    -s   &            & \textit{Anfang-s+zeit}           & beginning time   & `initial period' \\
    -n   & \Checkmark & \textit{Katze-n+pfote}           & cat paw          & `cat's paw' \\
    -en  & \Checkmark & \textit{Frau-en+stimme}          & woman voice      & `female voice' \\
    *e   &            & \textit{Kirsch*kuchen}           & cherry cake      & `cherry cake' \\
    -e   & \Checkmark & \textit{Geschenk-e+laden}        & gift store       & `gift store' \\
    =e   & \Checkmark & \textit{Händ=e+druck}            & hand press       & `handshake' \\
    =    & \Checkmark & \textit{Mütter=+zentrum}         & mother centre    & `centre for mothers' \\
    -er  & \Checkmark & \textit{Kind-er+buch}            & child book       & `children's book' \\
    =er  & \Checkmark & \textit{Büch=er+regal}           & book shelf       & `bookshelf' \\
    -ns  &            & \textit{Name-ns+schutz}          & name protection  & `name protection' \\
    -ens &            & \textit{Herz-ens+angelegenheit}  & heart matter     & `affair of the heart' \\
    \bottomrule
    \end{tabular}
  }
  \caption{Overview of the different linking elements in German nominal compounds; \textit{=} is used to denote linking elements which trigger umlaut on N1, \textit{*e} denotes linking elements where the final schwa of N1 is deleted; the column labelled $\equiv$\textit{P} indicates whether the linking element is formally identical to the plural of some declension class of N1}
  \label{tab:linkages}
\end{table}

Whether \textit{-en} and \textit{-n} or \textit{-ens} and \textit{-ns} should be described as allomorphic variants (\ie \textit{-(e)n} and \textit{-(e)ns}) is debated.
See Sections~3.2 and~3.6 of \textcite{NueblingSzczepaniak2013} for an argument in favour of this view and \textcite[33--36]{Neef2015} for an argument against it.
This issue does not affect our study, and we separate the two potential allo-forms in order to maximise the informativity of our results.
Since we do not look at linking elements with \textit{-s} and \textit{-es} in detail, however, we have conflated these two into \textit{-(e)s}.
This should not be equated with referring to the linking elements as allomorphs, which they are not, although the genitive suffixes from which they developed are \parencite[81]{Szczepaniak2016,NueblingSzczepaniak2013}.
This was done simply for ease of exposition, since \textit{-es} cannot be a plural suffix and, although \textit{-s} can be, it is avoided within compounds after N1s that take \textit{-s} as their plural suffix (cf.\ \citealt{Wegener2003,Wegener2005,NueblingSzczepaniak2013}).

The different linking elements have quite different frequencies, with the zero and \textit{-(e)s} linking elements being by far the most frequent ones.
For example, \textcite[177]{Gallmann1998} reports that $70\%$ of all N1+N2 compounds have the zero linking element (not specifying whether he refers to type or token frequency).
\textcite[29]{KrottEa2007} report that the compound types with the zero linking element make up $65\%$ of all compounds in the CELEX database.

Finally, it should be mentioned that in some terminologies, linking elements would not be called linking elements when they are actual plural markers (\ie when plural form of the linking element and plural meaning of N1 occur together; \citealt{Dressler1987}).
We follow the liberal choice of words by \textcite{BangaEa2013b}, who also call the respective affixes ``linking elements'' regardless of whether or not they mark plural meaning (see \citealt[196]{BangaEa2013b} for a discussion of stricter views).

\subsubsection{Conditions on linking element selection in German}
\label{sec:conditionsonlinkageselectioningerman}

The choice of linking element in any given compound is not fully predictable, but the grammatical gender of N1, its declension class (dependent largely on gender), its phonotactics, and possibly plural semantics all limit the options significantly (see \citealt{Fuhrhop1996,NueblingSzczepaniak2013}).
All these morpho-phono\-logical and lexical factors which partially determine the choice of linking element are based on N1, so the distribution of linking elements in general is unanimously treated as suffixation of N1.

Several (soft) descriptive generalisations concerning linking element distribution can be made (cf.\ e.g.\ \citealt{Fuhrhop1996} and \citealt{NueblingSzczepaniak2013} for extensive overviews).
Derived nouns ending in suffixes like \textit{-ung}, \textit{-heit}, \textit{-tum}, etc.\ have a strong tendency to occur with an \textit{-s} linking element, as in \textit{Heizung-s+wartung} `heating maintenance'.
Simplex masculine and neuter nouns are those which often take the \textit{-(e)s} linking element, in which case the form is identical to the genitive singular, for example \textit{Boot-s+fahrt} `boat trip' (with neuter \textit{Boot}) or \textit{Tag-es+form} `form of the day' (with masculine \textit{Tag}).
Also, N1s ending in full (non-reduced or tense) vowels such as \textit{Oma} `grandma' or \textit{Auto} `car' can never take the \textit{-s} linking element (although \textit{-s} is their only inflectional suffix when they are used as independent nouns) and occur with a zero linking element as in \textit{Auto+wäsche} `car wash'.%
\footnote{See \textcite{Wegener2003, Wegener2005} and \textcite{FuhrhopKuerschner2015} for further discussion of the \textit{-s} linking element and \textcite{Fehringer2009} for an investigation of an emerging use of \textit{-s} to mark plurality of N1 in northern German dialects.}
Furthermore, feminine nouns ending in schwa virtually always occur with the linking element \textit{-n} \parencite[32]{LibbenEa2002}, which is also their plural morpheme.
So-called weak masculine nouns and so-called mixed masculine and neuter nouns, which have an \textit{-(e)n} plural, often occur with an \textit{-(e)n} linking element, \egg \textit{Linguist-en+witz} `linguist's joke\slash joke about linguists' and \textit{Schwede-n+humor} `Swede humor'.%
\footnote{For weak nouns, the \textit{-(e)n} linking element is also identical to all non-nominative-singular forms, because they follow an otherwise unusual paradigm where the nominative singular is unmarked and all other forms are marked identically with \textit{-(e)n} (see \citealt{Koepcke1995,Schaefer2016c}).}
Finally, \textit{-ens} and \textit{-ns} are idiosyncratic and rare, used only with a handful of N1s.

More systematically, \textcite{DresslerEa2001} showed in a cloze test that the choices German native speakers make with respect to linking elements in novel compounds are partially predictable from rules, but that an analogical component (similarity to existing compounds with the same first constituent) is required.
The approach in \textcite{KrottEa2007} shows how a model with very high explanatory power can be constructed completely without rules or exceptions, based only on analogy.
Using an implemented exemplar-based model, experimental validation, and post-hoc analysis, they find that the choice of linking elements in novel compounds can be predicted well by considering exemplar families of compounds that have the same N1 together with features of the first constituent such as rime, gender, and inflectional class \parencite[47]{KrottEa2007}.

In sum, the zero linking element can be considered a sort of default, since it is the most frequent form of linking element in N1+N2 compounds and no compound is phonologically unacceptable with the zero linking element.
When a linking element does appear, which form it takes is affected primarily by N1.
N1s have certain preferences for linking element selection, and they also often disprefer particular linking elements (partially depending on gender, stem-final segments, and the derivational status of N1).
There is solid evidence that exemplar effects can account for most selection tendencies.
It remains a noteworthy fact, however, that except for the zero and \textit{-(e)s} linking elements, the linking elements which N1s can take are almost always identical to their plural markers.

\subsubsection{Alternations between linking elements}

Despite the existence of more or less firm constraints on the selection of linking elements, which linking element will be chosen in a given compound is by no means fully pre-determined.
Dating back to \textcite{Augst1975}, alternations between one or more different linking elements with the same N1 have been described.
Alternations with more than two alternatives are rare, however.
\textcite[134--135]{Augst1975} reports that among the $4,025$ N1s he examined, $390$ occurred with two linking elements, $31$ with three, and only eight with four.
Notice that the proportion of N1s occurring with two different linking elements is considerable at $9.7\%$.
However, Augst (given the lack of searchable corpora at the time) only analysed forms found in normative dictionaries, which list lexicalised forms and not productively formed compounds.
We propose that corpus analysis will likely reveal many more N1s occurring in at least a two-way alternation between a zero linking element and a pluralic linking element.

Some researchers are skeptical towards the idea of productive alternation.
Roughly forty years after Augst's study, \textcite[31]{NeefBorgwaldt2012} and \textcite[46]{Neef2015} suggest that for each N1, there is only one productive and uniquely determined linking element.
They treat all cases of alternation as lexicalised ``exceptions''.

When faced with an apparent counterexample -- the N1 \textit{Ohr} `ear', which \textcite[42]{NeefBorgwaldt2012} found to occur in the normative spelling dictionary \textit{Duden} \parencite{Duden2006} in 19 compound types with a pluralic linking element (\textit{Ohr-en}) and in 15 with no linking element -- they conclude not that these forms alternate, but that ``the method does not result in a clear picture about the productive form of the first constituent'' and that there might simply be a ``a developing allomorphy of the first constituent for this particular lexeme''.%
\footnote{``Damit liefert die vorgeschlagene Methodik kein klares Bild zur produktiven Vordergliedsstammform.
Denkbar ist allerdings, dass sich für dieses spezielle Lexem eine Allomorphie auf der Ebene der Vordergliedsstammform herausgebildet [\ldots]'' \parencite[31]{NeefBorgwaldt2012}.}
Although the authors refer to their approach as a ``corpus study'', most corpus linguists nowadays would very likely reserve the term ``corpus study'' for work using naturally produced linguistic data.
Dictionaries do not reflect productive language use (especially with highly productive phenomena such as compounding in German) but list lexicalised words by definition.
We present a corpus study in the more widely accepted sense in Section~\ref{sec:data}.


\subsection{Linking elements as plural markers}
\label{sec:linkagesaspluralmarkers}

\subsubsection{Plural interpretation of pluralic linking elements in German and Dutch}
\label{sec:pluralinterpretationofpluraliclinkagesingermananddutch}

Our study focusses on N1s which alternate between a pluralic and a non-pluralic linking element, and it is our goal to show that these linking elements actually have a plural interpretation and to clarify when this appears.
If they do have a plural interpretation, then at least in some compounds, the linking element is a case of plural inflection inside a compound.
We now review some of the research on a potential plural interpretation of pluralic linking elements.
We will first consider theoretical and empirical approaches to this question for German compounds, followed by the more advanced research on this question that has been conducted for Dutch.

An extreme theoretical position with respect to any functional or semantic interpretation of linking elements is espoused by \textcite{NeefBorgwaldt2012} and \textcite{Neef2015}.
In these papers, the authors state that linking elements cannot be functional, since no one proposed function (for instance, to provide a conceptual boundary between constituents, cf.\ \citealt[530]{Fuhrhop1996}; to close an open syllable ending on schwa, cf.\ \citealt[446]{Wegener2003}; to ease articulation, cf.\ \citealt[177]{Wegener2005}; or to act as a plural marker when N1 is conceptually plural, cf.\ \citealt[534]{Fuhrhop1996}, \citealt[427]{Wegener2003}) holds strictly for all linking elements across the board.
The plural marking function is not strict insofar as a pluralic linking element is not required for N1 to be interpreted as a semantic plural, and a pluralic linking element also does not reliably exclude a singular interpretation.
While this in no way excludes the possibility that a pluralic linking element is a soft cue for plurality, the authors seem to suggest that function\slash meaning and form have to stand in a one-to-one relationship, except for a small number of exceptions which have been lexicalised (\egg \citealt[42]{NeefBorgwaldt2012}).
They dismiss polyfunctional explanations altogether, because no stringent system of functions has been proposed which explains exactly under which conditions which linking element is chosen (\egg \citealt[27--29]{NeefBorgwaldt2012}).
Such a view (similar to the dual-route approach as discussed in the context of linking elements in \citealt{KrottEa2007}) has become less and less tenable in the face of recent research on the nature of the form--meaning relationship in morphology, which paints a clearly probabilistic picture where more often than not, similarity relations play a major role (see \citealt[107]{ArndtlappeEa2016}).
The present study adds to the evidence that such categorical approaches are inadequate with regard to the interpretation of German linking elements.

Theoretical approaches to the question of the possibility of inflection within compound boundaries also display conflicting perspectives.
For instance, \textcite[9]{Schluecker2012} states that there can be no inflection on non-heads in compounds, because these non-heads either do not inflect at all or the suffixes with which they occur (\ie linking elements) are non-inflectional by definition.
This perspective depends greatly on one's analysis of the characteristics of linking elements and on which type of inflection one considers.
While it is true that, for instance, linking elements no longer represent genitive case inflection like they originally did, we will show that linking elements should not be considered non-inflectional as concerns number.
This more liberal distinction is made by other scholars as well.
\textcite[577]{FuhrhopKuerschner2015}, for example, state in their survey of linking elements in Germanic languages that ``the expression of number by linking elements generally seems possible'', while case marking is mostly considered to be unavailable within compounds.
Furthermore, \textcite[178--180]{Gallmann1998} distinguishes internally licensed and externally licensed inflectional features.
According to him, externally licensed inflectional features are those which are determined by the syntagmatic context in which a noun occurs, such as case assignment by verbs.
Internally licensed features, on the other hand, are those which are assigned not by context, but by categorial membership (such as grammatical gender) or interpretation (such as plurality).
Gallmann maintains that externally licensed inflectional features -- prominently, case -- cannot be assigned to non-heads in compounds (N1s), but he does not explicitly exclude internally licensed features like plural marking on N1s.
Thus, the possibility of plural marking within German compounds remains theoretically open.

Interestingly, however, it was shown in \textcite{KoesterEa2004} that in the perception of spoken German compounds, hearers do not use linking elements with plural form as cues to plural semantics.
While this is a substantial finding in support of the impossibility of pluralic linking elements, the picture may very well be different for written German (much like \citealt{SchreuderEa1998} argue for effects of plural semantics in written Dutch compounds).

These examples, though far from all comments made on this topic, illustrate the divide among linguists and grammarians of German.
In spite of this, astonishingly little empirical research has been published on the question.
However, for Dutch, there exists a decades-long tradition of substantive experimental research into plural interpretations of pluralic linking elements.
\textcite{SchreuderEa1998} and \textcite{BangaEa2012} demonstrate how a change in the official Dutch orthography, which assimilated the linking element to the plural marker graphemically, fostered the plural interpretation of non-heads of compounds in written Dutch.
Also, \textcite{BangaEa2013a} show that a plural interpretation of N1s in novel Dutch compounds is positively linked to the occurrence of the optional linking element \mbox{\textit{-en}}, which is homophonous with the plural marker.
First, they show that subjects prefer to use \textit{-en} with N1 in contexts where a plural meaning of N1 was made clear.
Second, they demonstrate that the preference for the pluralic linking element is even activated when the compound contains a singular form of N1 but when it is given a plural meaning by the context, which is evidence that the connection between the linking element and plurality is not just based on a formal recency effect.
They confirm that plural interpretation indeed creates a preference for using \textit{-en}, but that form-based repetition effects strengthen the meaning-based effect \parencite[45]{BangaEa2013a}.

Third and of special relevance to our study, they also test German L2 learners of Dutch and find that for them, the plural effect is weaker than for native speakers of Dutch.
They offer several possible interpretations for this \parencite[45--47]{BangaEa2013a}.
Citing \textcite{LibbenEa2002}, who show that linking elements in German come with a processing overhead, they speculate that German speakers might have a tendency to avoid linking elements.
Also, they propose that since German plurals are not always marked (there is a zero plural in German, just as there is a zero linking element), German speakers might associate linking elements less strongly with plurality.
Finally, they argue that plural markers in German are sometimes formally identical to case markers, which could also weaken the connection German speakers establish between plurality and linking elements.
We return to these ideas in Section~\ref{sec:conclusion} after having presented our own results.

In \textcite{BangaEa2013b}, the authors compare the interpretation of Dutch compounds by native speakers of Dutch to the interpretation of conceptually identical English compounds by English native speakers.
They compare plurality ratings for conceptually plural N1s, \ie those where a plural meaning of N1 is a de-facto necessity, to plurality ratings for conceptual non-plurals.
Their examples include \textit{bananenschil} `banana peel' for a conceptual non-plural and \textit{aardbeienjam} `strawberry jam' for a conceptual plural.
In essence, they show that Dutch speakers produce higher plurality ratings for compounds with \textit{-en} and lower plurality ratings for compounds without \textit{-en} for both groups of compounds (conceptually plural and non-plural) compared to English native speakers for the English compounds identical in meaning and structure.
That is, the linking element provides Dutch speakers with an additional cue for plurality, independent of the cues coming from the conceptual structure of the compound.
In a second experiment, they show that Dutch subjects as L2 speakers of English react to English compounds by and large like English speakers (for a report of some complications, see \citealt[211]{BangaEa2013b}).
This corroborates findings that the actual presence of the linking element is a key cue for plural interpretation for Dutch speakers.


\subsubsection{Consequences of plural linking elements for morphological theories}
\label{sec:consequencesofplurallinkagesformorphologicaltheories}

A crucial point in the discussion of inflection within compounds is whether frameworks and theories can deal with it appropriately, as we mentioned above.
In theories ranging from the strict layering hypothesis by \textcite{Siegel1979} through the Lexical Phonology of \textcite{Mohanan1986} and the A-Morphous Morphology by \textcite{Anderson1992} to the words-and-rules theory by \textcite{Pinker1999}, it has been maintained that inflection applies after derivational word formation (including compounding), leading to inflectional affixes being positioned farther toward the edges of words than derivational affixes and to inflectional affixes not occurring between constituents of compounds.
This was encoded in the respective theories as hard and putatively universal constraints.
Such approaches were disputed, however, from very early on, for example in \textcite{Bochner1984}, where the author shows how inflection can occur within derivation (see \citealt[2--3]{KirchnerNicoladis2009} for more on this debate).
Most pertinent in the context of our study is the argument in \textcite[47--48]{BangaEa2013a} that such restrictive theories fail to explain how the Dutch linking element \textit{-en} can have a plural interpretation (thus being an inflectional suffix) and occur between constituents of compounds (see their results summarised above).
The same would be true for German compounds if pluralic linking elements turn out to be plural markers systematically.

However, as \textcite[5]{KirchnerNicoladis2009} correctly point out, strong tendencies (falsely interpreted as universal constraints in older generative morphology) for inflection to appear only at the edges of words should still be explained.
Given this goal, we consider it highly instrumental to describe exactly where and when inflection does \textit{not} appear at the edges of words, and this paper is about such a case. %(see Section~\ref{sec:conclusion}).
As we have argued in Section~\ref{sec:linkingelementsinprobabilisticmorphology}, frameworks should not impose unnecessary restrictions which stand in the way of describing phenomena as they actually occur.
Furthermore, the inherent probabilistic nature of linguistic phenomena -- most obvious in situations where more than one option is available in a so-called \textit{alternation} -- means that investigating preferences might be more useful than searching for strict rules and lexicalised exceptions.

\subsection{Outline of our study}

Our main hypothesis under investigation is whether a pluralic linking element in a N1+N2 compound is systematically associated with a semantic plural by native speakers of German.
We use the term ``systematic'' in the sense of probabilistic grammar, where preferences (under specific given conditions) are not assumed to be binary but probabilistic, weighted, and better described as numerical rather than discrete.
As sources of data, we will use a large corpus and an experimental setup.
Since annotating corpus exemplars reliably for whether the N1 in a compound has a plural meaning is quite difficult in the general case, we isolate two specific and relatively easy-to-detect configurations in which a pluralic linking element might signal plurality, where the second configuration (internal plural) is the one we consider to be the crucial one in showing that pluralic linking elements are associated with plural meaning:

\vspace{\baselineskip}

\begin{enumerate}
  \item A plural on the entire compound (formally on the head constituent) might trigger the use of a pluralic linking element.
  We call this the \textit{external plural effect}.
  \item Certain semantic classes of N2s standing in an appropriate semantic relation with N1 might force N1 to have a plural interpretation and therefore lead to a preference for using the pluralic linking element.
  We call this the \textit{internal plural effect}.
\end{enumerate}

\vspace{\baselineskip}

The external plural effect on linking element selection might have two different motivations.
First, there are cases where the referents of N1 necessarily form a set or sum entity with more than one member when the compound as a whole is a plural.
This might lead to a preference of pluralic linking elements.
For example, we might see a preference for a compound \textit{Hund+herz} `dog's heart' with the non-pluralic linking element (zero in this case) in the singular but \textit{Hund-e+herzen} `dogs' hearts' with the pluralic linking element (\textit{-e} in this case) in the plural.
It is not necessary to distinguish between different non-pluralic linking elements (mainly zero and \textit{-(e)s}).
This effect depends to a large extent on the semantics of both constituents and the compound.
In the case of \textit{Hundherz}, the effect is clear because each dog has exactly one heart, so multiple dogs entail multiple hearts.
With other compounds, such as \textit{Brot+mahlzeit} or \textit{Brot-e+mahlzeit} `bread meal', the picture is blurrier because a single loaf or piece of bread can make several meals, and more than one piece or loaf of bread can be consumed in one meal.
Making matters worse, it could be the generic or mass noun meaning of \textit{Brot} which is addressed, in which case plurality does not even make sense.
Since we found that in a majority of the cases, these ambiguities cannot be resolved and the sentences are compatible with both interpretations, we treat the external plural effect on a strictly formal level.

The second potential motivation for the external plural effect is a purely formal one; there might simply be plural agreement within the compound.
\textcite{BangaEa2013a} and \textcite{BangaEa2013b} have found that there are effects related to the mere presence of a formal plural of N1 in the context of the compound.
However, the plural on the whole compound might trigger a preference for a pluralic linking element on N1, even if the semantic motivation for the external plural effect does not apply.
Thus, even if we find evidence for an external plural effect, we could not be sure that it is an effect related to plural meaning.

The internal plural effect, on the other hand, is only related to the lexical meanings of the compound's constituents, and it is similar to the conceptual plurals in \textcite{BangaEa2013b}, as discussed above.
Prominently, N2s that provoke this effect might have a collective meaning.
In this case, regardless of the grammatical number of the whole compound, there are necessarily several referents of N1 involved conceptually.
Examples include true collectives like \textit{Kindergruppe} `group of children', metaphorical collectives as in \textit{Zitateregen} `rain of quotations', reciprocals such as \textit{Räderwechsel} `swapping of tyres', or relational N2s as in \textit{Lochdistanz} `distance between (the) holes'.

The external plural condition is weaker both conceptually and in terms of operationalisation, so if pluralic linking elements are indeed interpretable as plural markers, we expect to find an effect especially for the internal plural condition.
Interestingly, if we find evidence only for the internal plural effect but not the external plural effect, this would fit within a probabilistic version of previous findings that plurality in compounds is purely conceptual or inherent to N1 and does not depend on the grammatical context of the compound (see discussion above, for example \citealt{Gallmann1998}).

In Section~\ref{sec:corpusstudy}, we study these two specific hypotheses using corpus data.
Then, in Section~\ref{sec:split100experiment}, we examine them in an experimental paradigm (split-100 ratings).
Before we turn to these studies, Section~\ref{sec:data} describes the exploratory work that allowed us to make an informed selection of items for the studies.
It also shows that there is a significant number of N1s alternating between pluralic and non-pluralic linking elements, contrary to assumptions made by some researchers such as \textcite{NeefBorgwaldt2012,Neef2015}.
