\section{Linking elements in German}
\label{sec:linkingelementsingerman}

\subsection{The form and distribution of linking elements in German}
\label{sec:theformanddistributionoflinkingelementsingerman}

\subsubsection{German linkages}

In this paper, we exclusively deal with determinative (endocentric) nominal compounds, \ie compounds formed through the concatenation of two constituent nouns (possibly with intervening linking elements, see below) where the second noun (N2) is the head modified by the first noun (N1).
While compounding is a recursive process in principle, we ignore complications involved with more complex structures and look only at binary compounds.
We refer to these structures as \textit{N1+N2 compounds}.

In a majority of such compounds in German, N1 and N2 are simply concatenated, and N1 appears in its base (uninflected) form, which is identical to at least the nominative singular form.
A linking element (LE) is optional segmental material inserted between N1 and N2.
LEs in compounds are not exclusive to German; see \textcite{SchreuderEa1998,BangaEa2013a,BangaEa2013b} on plural linking elements in Dutch.%
\footnote{We resist the temptation of equating structures across languages too hastily and restrict our review to Germanic languages with highly similar systems.
For example, in \textcite[4]{KirchnerNicoladis2009} (citing \citealt{Spencer1991}, page numbers not specified), it is claimed that in Turkish ``a variety of inflectional suffixes can appear within compounds'', including (according to the glosses) a possessive suffix \textit{-i} and a putative person marker \mbox{\textit{-um}}.
They cite an example \textit{vurd-um duy-maz} which is usually spelled \textit{vurdumduymaz}.
First, the appropriate segmentation would be \textit{vur-du-m} with \textit{vur} `hit', \textit{-du} PRT, and \textit{-m} \textsc{1SG}.
It is a chunk fossilised as a word with an original literal meaning of roughly `I hit, he doesn't feel', now meaning `thick-skinned'.
Furthermore, \textcite[313--319]{Spencer1991} mainly discusses the izafet construction (with its marker \textit{-i}), which is arguably more phrasal than word formation processes, embedded in units that resemble compounds.
All the examples are thus very different from Germanic N1+N2 compounds with linking elements and might be more similar to compounds with phrasal first constituents such as \textit{do or die situation} (see \citealt[366]{Olsen2015}), where surely different regularities apply.
}
However, German speakers use an unusually high number of different LEs, and they also use other types of linkage.
The list of segmental LEs is widely accepted to be: \textit{e}, \textit{er}, \textit{s}, \textit{es}, \textit{n}, \textit{en}, \textit{ns}, \textit{ens} (see \citealt[31]{Neef2015}, \citealt{KrottEa2007}).
Also, \textit{e} and \textit{er} come with an optional umlaut on N1, and some compounds just have umlaut on N1 and no segmental LE at all.
Last, N1s ending in schwa (graphemically \textit{<e>}) can drop this schwa in compounds.

In concrete analyses, we do not mark zero linkages at all.
Furthermore, \textit{-X} is the notation for a linking element \textit{X} attaching to N1, and \textit{=X} denotes a linking element attaching to N1 with additional umlaut.
An umlaut-only linkage is indicated by \textit{=} alone.
The special notation \textit{*e} is used for a linkage where the final schwa of N1 is deleted.
Furthermore, \textit{+} separates N1 (possibly with linking element) and N2 in our analyses.
Graphemically, however, German compounds are usually spelled as one word.%
\footnote{Some compounds, for example those involving names or loan words, are sometimes written with a hyphen, such as \textit{Adenauer-Zeit} `Adenauer period'.
We excude them from our study because they virtually never occur with LEs and have specific compound-internal semantics.
There are also occasional spellings of compounds as two words such as \textit{Blumen Welt} `world of flowers' and with in-word uppercase letters such as \textit{BlumenWelt} which are much debated from a normative perspective (see also \citealt{Scherer2012}).
These spellings are assumed to be specific of certain genres (such as adverts) and are quite rare overall.
Therfore, we do not include them in our study.}
Table~\ref{tab:linkages} illustrates the different types of linkages in German.%
\footnote{Different types of linkages are also found in other types of German compounds, for example N+A compounds such as \textit{herzensgut} (\textit{Herz-ens+gut}) `kindhearted' or \textit{lebensfreundlich} (\textit{Leben-s+freundlich}) `life-sustaining\slash livable' or \textit{hundelieb} (\textit{Hund-e+lieb} `dog-loving'.
While their distribution, function, and interpretation in such compounds might be related to that in N1+N2 compounds, our studies reported in Sections~\ref{sec:corpusstudy} and~\ref{sec:split100experiment} make use of diagnostic features exclusive to N1+N2 compounds, and we therefore excluded all other types of compounds.}

\begin{table}[htbp]
  \resizebox{\textwidth}{!}{
    \begin{tabular}{lllll}
    \toprule
    Linkage & Pl & example & literal gloss & translation \\
    \midrule
    ∅    &            & \textit{Haus-tür}                & house door       & `front door' \\
    -s   &            & \textit{Anfang-s+zeit}           & beginning time   & `initial period' \\
    -n   & \Checkmark & \textit{Katze-n+pfote}           & cat paw          & `cat's paw' \\
    -en  & \Checkmark & \textit{Frau-en+stimme}          & woman voice      & `female voice' \\
    *e   &            & \textit{Kirsch+kuchen}           & cherry cake      & `cherry cake' \\
    -e   & \Checkmark & \textit{Geschenk-e+laden}        & gift store       & `gift store' \\
    =e   & \Checkmark & \textit{Händ=e+druck}            & hand press       & `handshake' \\
    =    & \Checkmark & \textit{Mütter=+zentrum}         & mother centre    & `centre for mothers' \\
    -er  & \Checkmark & \textit{Kind-er+buch}            & child book       & `children's book' \\
    =er  & \Checkmark & \textit{Büch=er+regal}           & book shelf       & `bookshelf' \\
    -ns  &            & \textit{Name-ns+schutz}          & name protection  & `name protection' \\
    -ens &            & \textit{Herz-ens+angelegenheit}  & heart matter     & `affair of the heart' \\
    \bottomrule
    \end{tabular}
  }
  \caption{Overview of the different linkages in German nominal compounds; \textit{=} is used to denote linking elements which trigger umlaut on N1, \textit{*e} denotes linkages where the final schwa of N1 is deleted; the column labelled \textit{Pl} indicates whether the linkage is (in most cases) formally identical to the plural of N1 (see below)}
  \label{tab:linkages}
\end{table}

Whether \textit{-en} and \textit{-n} and \textit{-ens} and \textit{-ns} should be described as allomorphic variants is debated.
See Sections~3.2 and~3.6 of \textcite{NueblingSzczepaniak2013} for an argument in favour of this view and \textcite[33--36]{Neef2015} for an argument against it.
This issue does not affect our study, and we separate the two potential allo-forms in order to maximise the informativity of our results.
Since we do not look at \textit{-(e)s} linkages in detail, however, we conflated these two.
The different linkages have quite different frequencies, with zero and \textit{-(e)s} linkages being by far the most frequent ones.
For example, \textcite[177]{Gallmann1998} reports that $70\%$ of all N1+N2 compounds have zero linkage, not specifying whether he refers to type or token frequency.
\textcite[29]{KrottEa2007} report that the compound types with zero linkage make up for $65\%$ of all compounds in the CELEX database.
See Section~ \ref{sec:databaseandproductivityassessment} for a detailed breakdown of the type and token frequencies in the 21 billion token DECOW16A corpus.

\subsubsection{Conditions on linkage selection in German}

The choice of linking element is not fully predictable, but the grammatical gender of N1, its declension class (which depends largely on gender), the phonotactics of N1, and possibly plural semantics limit the options severely (see \citealt{Fuhrhop1996,NueblingSzczepaniak2013}).
All these morpho-phono\-logical and lexical factors, which partially determine the choice of linkage, relate strictly to N1, such that the distribution of linking elements and linkages in general is unanimously treated as suffixation of N1.
For example, derived nouns ending in suffixes like \textit{-ung}, \textit{-heit}, \textit{-tum}, etc.\ have a trong tendency to occur with an \textit{-s} linkage as in \textit{Heizung-s+wartung} `heating maintenance'.
Simplex masculine and neuter nouns also optionally take the \textit{-(e)s} LE, in which case the form is identical to the genitive singular, for example \textit{Boot-s+fahrt} `boat trip' (with neuter \textit{Boot}) or \textit{Tag-es+form} `form of the day' (with masculine \textit{Tag}).
N1s ending in full (non-reduced or tense) vowels such as \textit{Oma} `grandma' or \textit{Auto} `car' can never take the \textit{-s} LE (although \textit{-s} is their only inflectional suffix when they are used as independent nouns) and tend to occur with a zero linkage as in \textit{Auto+wäsche} `car wash'.
See \textcite{Wegener2003}, \textcite{FuhrhopKuerschner2015} for many details on \textit{-s} linkages.

Feminine nouns ending in schwa virtually always occur with an \textit{-n} linkage \parencite[32]{LibbenEa2002}, which is also their plural morpheme.
So-called weak masculine nouns and so-called mixed masculine and neuter nouns, which have an \textit{-(e)n} plural, often occur with an \textit{-(e)n} LE, for example \textit{Linguist-en+witz} `linguist's joke' or `joke about linguists' and \textit{Schwede-n+humor} `Swede humor'.%
\footnote{For weak nouns, the \textit{-(e)n} LE is also identical to all non-nominative singular forms because they follow an otherwise unusual paradigm where the nominative singular is unmarked and all other forms are marked identically with \textit{-(e)n} (see \citealt{Koepcke1995,Schaefer2016c}).}
Finally, \textit{-ens} and \textit{-ns} are idiosyncratic and only used with a handful of N1s.
In sum, while the zero linkage is some kind of default, N1s from different declension classes (often partially predetermined by gender, stem-final segements, and the derivational status of N1) have other preferences and are often blocked from occurring with certain linkages.
Beyond the zero and \textit{-(e)s} linkages, the linking elements which N1s can take are mostly identical to their plural markers.


\subsubsection{Alternations between linkages}

Despite the existence of more or less hard constraints on the selection of linkages, it is by no means fully pre-determined which linkage will be chosen in a given compound.
Dating back to \textcite{Augst1975}, alternations of N1s between one or more different linkages have been dscribed.
Alternations with more than two alternatives are apparently rare, however.
\textcite[134--135]{Augst1975} reports that among the $4,025$ N1s examined by him, $390$ occurred with two linkages, $31$ with three, and only $8$ with four.
Notice that the number of N1s occurring with two different linkages is considerable at $9.7\%$.
We propose that corpora containing more liberal usage of language which is not strongly bound by norms combined with contemporary possibilities of large-scale automatic analysis will reveal many more N1s occurring in at least a two-way alternation between a zero linkage and a pluralic linkage (see Section~\ref{sec:data} and Section~\ref{sec:corpusstudy}).

Some researchers are skeptical towards a general alternation, however.
Roughly forty years after Augst's study, \textcite[31]{NeefBorgwaldt2012} and \textcite[46]{Neef2015} seem to suggest that for each N1, there is a uniquely determined linkage, and they treat all cases of alternation as lexicalised ``exceptions''.
In \textcite[42]{NeefBorgwaldt2012}, the authors make what they call a ``corpus study'' based on the normative spelling dictionary \textcite{Duden2006}.
Based on $19$ found compounds containing \textit{Ohr} `ear' as N1 with a pluralic linkage (\textit{Ohr-en}) and $15$ with a non-pluralic (zero) link, they conclude that ``the method does not result in a clear picture about the productive form of the first constituent'' and that there might simply be a ``a developing allomorphy of the first constituent for this particular lexeme''.%
\footnote{``Damit liefert die vorgeschlagene Methodik kein klares Bild zur produktiven Vordergliedsstammform.
Denkbar ist allerdings, dass sich für dieses spezielle Lexem eine Allomorphie auf der Ebene der Vordergliedsstammform herausgebildet [\ldots]''}
In the light of contemporary methods in corpus linguistics, such an argumentation can be dismissed on purely methodological grounds.
We describe a principled approach to assessing productive linkage alternations for N1s in Section~\ref{sec:data}.


\subsection{Linkages as plural markers}
\label{sec:linkagesaspluralmarkers}

\subsubsection{Plural interpretation of pluralic linkages in German and Dutch}

Our study focusses on N1s which alternate between a pluralic and a non-pluralic linkage, and it is our goal to show that (and when) they actually have a plural interpretation.
If they do have a plural interpretation, then at least in some compounds, the linkage is a case of plural inflection inside a compound.
We are not aware of any systematic corpus study or experimental work on this for German, but many researchers have made statements based on their theoretical views and the general descriptive consensus.

A rather extreme position with respect to functional interpretations of linkages is taken by \textcite{NeefBorgwaldt2012} and \textcite{Neef2015}.
In these papers, the authors focus on the (undisputed) fact that none of the functions attributed to different linkages (see Section~\ref{sec:theformanddistributionoflinkingelementsingerman}), including a potential function as a plural marker, applies strictly.
Especially the plural marking function is not strict in that a pluralic linkage is required for N1 to be interpreted as a semantic plural.
Except for a small number of lexicalised ``exceptions'' (\egg \citealt[42]{NeefBorgwaldt2012}), the authors seem suggest that function and form have to stand in a one-to-one relationship.
The authors dismiss polyfunctional (and probably multifactorial) explanations because, as they say, no stringent system has been proposed which explains exactly under which conditions which linkage is chosen (\egg \citealt[27--29]{NeefBorgwaldt2012}).

\textcite[9]{Schluecker2012}, summing up some previous research, states that there can be no inflection on non-heads in compounds because these non-heads either does not inflect at all anyway or the suffixes with which they occur are not inflectional by definition.
In their survey of linking elements in Germanic languages, \textcite[577]{FuhrhopKuerschner2015} state more liberally that ``the expression of number by linking elements generally seems possible'' while case marking is mostly considered to be unavailable within compounds.

\textcite[178--180]{Gallmann1998} distinguishes between internally licensed and externally licensed inflectional features.
While he adopts a specific formulation within the Government and Binding framework, we focus on the gist of his generalisation, which can be expressed well outside of that specific framework.
Externally licensed inflectional features are those which are determined by the syntagmatic context in which a noun occurs, such as case assignment by verbs.
Internally licensed features, on the other hand, are those which are assigned not by context, but by categorial membership (such as grammatical gender) or interpretation (such as free plurals).%
\footnote{An example of a non-free plural according to \textcite[179]{Gallmann1998} would be the plural on \textit{students} in \textit{Dale and Daryl are students}.
See his parallel German example (4b).
}
Gallmann maintains that externally licensed inflectional features -- prominently, case -- cannot be assigned to non-heads in compounds (N1s), but he does not explicitly exclude plural marking on N1s.

These examples, although far from complete, illustrate the divide among linguists and grammarians of German.
Interestingly, no strong proponents of an interpretation of pluralic linkages as plural markers have so far weighed in, and no systematic empirical studies have been published.
However, for Dutch, there exists a decade-long tradition of substantive experimental research into plural interpretations of pluralic linkages.
\textcite{SchreuderEa1998} and \textcite{BangaEa2012} demonstrate how a change in the official Dutch orthography, which assimilated the linking element to the plural marker graphemically fostered the plural interpretation of the non-head in compounds.
Also, \textcite{BangaEa2013a} show that plural interpretation for N1s in novel Dutch compounds is positively linked to the occurrence of the optional linking element \textit{-en}, which is homophonous with the plural marker.
First, they showed that subjects prefer to use \textit{-en} with N1 in contexts where a plural meaning of N1 was made clear.
Second, they demonstrate that the preference for the pluralic linkage is also activated when the context addresses a plural meaning of N1 but contains a singular form of N1, which is evidence that the connection between the linking element and plurality is not just based on a formal recency effect.
They confirm that plural interpretation indeed creates a preference for using \textit{-en}, but that form-based repetition effects strengthen the meaning-based effect \parencite[45]{BangaEa2013a}.
Finally, they also test German L2 learners of Dutch and find that for them, the effect is weaker than for native speakers of Dutch.
They offer two possible interpretations for this \parencite[45--47]{BangaEa2013a}.
Citing \textcite{LibbenEa2002}, who show that linking elements in German come with a processing overhead, they speculate that German speakers might have a tendency to avoid linking elements.
Also, they propose that the fact that German plurals are not always marked, German speakers might associate LEs less strongly with plurality.
Finally, they argue that plural markers in German are sometimes formally identical to case markers, which could also weaken the connection German speakers make between plurality and linking elements.
We return to these ideas in Section~\ref{sec:functionandinterpretation} after having presented our own results.

In \textcite{BangaEa2013b}, the authors compare the interpretation of Dutch compounds by native speakers of Dutch to the interpretation of conceptually identical English compounds by English native speakers.
They compare plurality ratings for conceptually plural N1s, \ie those where a plural meaning of N1 is a de-facto necessity, to plurality ratings for coneptual non-plurals.
Their examples include \textit{bananenschil} `banana peel' for a conceptual non-plural and \textit{aardbeienjam} `strawberry jam' for a conceptual plural.
In essence, they show that Dutch speakers produce higher plurality ratings for compounds with \textit{-en} and lower plurality ratings for compounds without \textit{-en} for both groups of compounds (conceptually plural and non-plural) compared to English native speakers for the English compounds identical in meaning and structure.
That is, the linking element provides Dutch speakers with an additional cue for plurality, independent of the cues coming from the conceptual structure of the compound.
In a second experiment, they then show that Dutch subjects as L2 speakers of English react to English compounds by and large like English speakers (some complications aside, see \citealt[211]{BangaEa2013b}).
This corroborates findings that the actual presence of the linking element is a key cue for plural interpretation for Dutch speakers.

Finally, it should be mentioned that in some terminologies, LEs would not be called LEs when they are actual plural markers,
We follow the liberal choice of words by \textcite{BangaEa2013b}, who also call the respective affixes ``linking elements'' when they mark plural meaning (see \citealt[196]{BangaEa2013b} of a discussion of stricter views).


\subsubsection{Consequences for theories}

A crucial point in the discussion of inflection within compounds is whether frameworks and theories can deal with it appropriately.
From the \textit{Strict Layering Hypothesis} by \textcite{Siegel1979} over the \textit{Lexical Phonology} of \textcite{Mohanan1986} to the \textit{Words-and-Rules Theory} by \textcite{Pinker1999}, it was maintained that inflection applies after word formation (including compounding) derivationally, leading to inflectional affixes being positioned farther at the edges of words than derivational affixes and to inflectional affixes not occurring between constituents of compounds.
This was encoded in the respective theories as hard and putatively universal constraints.
Such approaches were disputed, however, from very early on, for example in \textcite{Bochner1984}, where the author shows how inflection can occur within derivation (see \citealt[2--3]{KirchnerNicoladis2009} for more on the historic debate).
Most pertinent in the context of our study is the argument in \textcite[47--48]{BangaEa2013a} that such restrictive theories fail to explain how the Dutch linking element \textit{-en} can have a plural interpretation (thus being an inflectional suffix) and occur between constiuents of compounds (see their results summarised above).
The same would be true for German compounds if pluralic linkages turn out to be plural markers systematically.

However, as \textcite[5]{KirchnerNicoladis2009} correctly point out, strong \textit{tendencies} (falsely interpreted as universal constraints in older generative morphology) for inflection to appear only at the edges of words still should be explained.
Given this goal, we consider it highly instrumental to describe exactly where and when inflection does \textit{not} appear at the edges of words, and this paper is about such a case (see Section~\ref{sec:functionandinterpretation}).
As we have argued in Section~\ref{sec:probabilisticgrammar}, frameworks should not impose unnecessary restrictions which stand in the way of describing phenomena as they actually occur.
Furthermore, the inherent probabilistic nature of linguistic phenomena -- most obvious in situations where more than one option is available in a so-called \textit{alternation} -- means that investigating preferences might be more adequate than searching for strict rules and lexicalised exceptions.

\subsubsection{Outline of our study}

Our main hypothesis under investigation is whether a pluralic linkage in a N1+N2 compound is systemtically interpreted as a semantic plural by native speakers of German.
We use the term ``systematic'' in the sense of Probabilistic Grammar, where preferences under specific conditions are not discrete but probabilistic, weighted, and better described as numerical rather than discrete.
As sources of data, we will use a large corpus and an experimental setup.
Since annotating corpus exemplars for whether the N1 in a compound has a plural meaning is quite difficult in the general case, we isolate two specific and relatively easy to detect configurations in which a pluralic linking element might signal semantic plurality:

\vspace{\baselineskip}

\begin{enumerate}
  \item A plural on the entire compound (formally on the head constituent) might trigger the use of a pluralic linkage.
  We call this the \textbf{external plural effect}.
  \item Certain N2s combined with an appropriate semantic relation between N1 and N2 might force N1 to have a plural interpretation and therefore lead to a preference for using the pluralic linkage.
  We call this the \textbf{internal plural effect}.
\end{enumerate}

\vspace{\baselineskip}

The external plural effect on linkage selection might have two different motivations.
First, there are cases where necessarily the referents of N1 form a set or sum entity with more than one member when the compound is a plural.
This might lead to a preference of pluralic linkages.
For example, we might see a preference for a compound \textit{Hund+herz} `dog's heart' with the non-pluralic linkage (zero on this case) in the singular but \textit{Hund-e+herzen} `dogs' hearts' with the pluralic linkage (\textit{-e} in this case) in the plural.
For this, it is not necessary to distinguish between different non-pluralic linkages (mainly zero and \textit{-(e)s}).
However, this effect depends to a large extent on the semantics and is hard to annotate.
In the case of \textit{Hundherz} the effect is clear because each dog only has one tail.
With other compounds, such as \textit{Brot+mahlzeit} or \textit{Brot=e+mahlzeit} `bread meal', matters become blurry because a single loaf of piece of bread can make several meals, and more than one piece or loaf of bread can be consumed in one meal.
Making matters worse, it could be the generic\slash mass noun meaning of \textit{Brot} which is addressed, in which case plurality
However, a second motivation for the external plural effect might be a simple plural agreement within the compound.
\textcite{BangaEa2013a} and \textcite{BangaEa2013b} have found that there are effects related to the mere presence of a formal plural of N1 in the context of the compound.
Related but slightly different, the plural on the whole compound might trigger a preference for a pluralic linkage on N1, even if the semantic motivation for the external plural effect does not apply.

The internal plural effect, on the other hand, is only related to the lexical meanings of the compound's constituents, and it is similar to the conceptional plurals in \textcite{BangaEa2013b} as discussed above.
There are cases where the semantics of N2 are more or less clearly collective.
In this case, regardless of the number on the whole compound, there are necessarily several referents of N1 involved on the conceptual side.
Examples contain true collectives \textit{Kindergruppe} `group of children', metaphorical collectives as in \textit{Zitateregen} `rain of quotations', reciprocals such as \textit{Räderwechsel} `swapping of tyres', or relational N2s as in \textit{Lochdistanz} `distance between (the) holes'.
While the external plural condition is weaker both conceptually and in terms of operationalisation, we should at least expect an effect for the internal plural condition if pluralic linkages are indeed interpretable as plural markers.
Interestingly, if we find evidence only for the internal plural effect but not the externalk plural effect, this would fit within a probabilistic version of previous findings that plurality in compounds is purely conceptual or inherent to N1 and nothing that depends on its grammatical context (see discussion above, for example \citealt{Gallmann1998}).

In Section~\ref{sec:corpusstudy}, we study these two more specific hypotheses using corpus data.
Then, in Section~\ref{sec:split100experiment}, we examine them in an experimental paradigm (split-100 ratings).
Before we turn to these studies, Section~\ref{sec:data} describes the exploratory work that allowed us to make a reasonable selection of items for the studies.
It also shows that there is a significant number of N1s alternating between a pluralic and a non-pluralic link, contrary to statements to the opposite in the literature (most prominently \citealt{NeefBorgwaldt2012,Neef2015}).
