\section{Linking elements in German}
\label{sec:linkingelementsingerman}


Therefore, the main hypothesis which we investigate is whether a pluralic link in a N1+N2 compound can be interpreted as a semantic plural by native speakers of German.
Since annotating corpus exemplars for whether the N1 in a compound has a plural meaning is quite difficult in the general case, we isolate two specific and easy to detect configurations in which a pluralic linking element might signal semantic plurality:
(i) a plural on the entire compound might trigger a plural interpretation of N1 and subsequently the use of a PL (external plural);
(ii) certain N2s combined with an appropriate semantic relation between N1 and N2 might force N1 to have a plural interpretation and therefore lead to a preference for using the PL (internal plural).
In Section~\ref{sec:corpusstudy}, we study these two more specific hypotheses using corpus data.
Then, in Section~\ref{sec:split100experiment}, we examine them in an experimental paradigm (split-100 ratings).
Before we turn to these studies, Section~\ref{sec:data} describes the exploratory work that allowed us to make a reasonable selection of items for the studies.
