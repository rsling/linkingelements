\section{Function and interpretation}
\label{sec:functionandinterpretation}

ONLY ISOLATED FRAGMENTS

As \parencite[365--366]{Olsen2015} puts it, ``compounds do not overtly express the grammatical or conceptual relations that exist between their constituents and are, consequently, inherently ambiguous'', but that ``compounds with plural and phrasal first constituents are not uncommon''.

This is, in our view, distinct from the question of whether non-head constituents in compounds can establish referents of their own, which \textcite[366]{Olsen2015} denies but \textcite{Dressler1987} demonstrates is possible.


Regardless of their diachronic development, speakers use pluralic linking elements as cues to the interpretation of N1+N2 compounds (see also \citealt[212]{BangaEa2013b}).


Note: Some compounds CANNOT even take a PL.
This stands in the way of an obligatory use of PLs in compounds where N1 is pluralic in meaning.
