\section{Conclusion}
\label{sec:conclusion}

\subsection{Ist it surprising that reader-writers use PL for plural semantics?}

Both in the corpus study (Section~\ref{sec:corpusstudy}) and the split-100 experiment (Section~\ref{sec:split100experiment}), we found evidence that reader-writers prefer to use a PL when N1 necessarily has a plural interpretation due to compound-internal semantics.
We found no evidence for a similar influence of a plural on the whole compound (external plural effect), which could have been semantically motivated in some cases or a simple compound-internal number agreement tendency.
Researchers have been suspecting this (see Section~\ref{sec:pluralinterpretationofpluraliclinkagesingermananddutch}), and in light of the previous research on Dutch LEs, this does not come as a surprise.
Regardless of their diachronic development, speakers use pluralic linking elements as cues to the interpretation of N1+N2 compounds (see also \citealt[212]{BangaEa2013b}).
Because the form of an N1 with a PL is one which reader-writers associate strongly with semantic plurals, they do so also when that form occurs in a compound.
A detailed look at our results shows reveals more about the actual mechanisms at work.

First, the effect strengths in the corpus study were weak to intermediate, and while the observed effect was stronger in the split-100 experiment, we are nowhere near a categorical split, and there is substantial per-speaker variation and uncertainty.
Readers should revisit Figure~\ref{fig:conditions} and Figure~\ref{fig:participants} to convince themselves that this is the case.
As pointed out in Section~\ref{sec:pluralinterpretationofpluraliclinkagesingermananddutch}, \textcite{LibbenEa2002} \hl{[Correctreference? Check!]} have found that there is a general tendency to avoid LEs, which also accounts for the high percentage of zero linkages (see Section~\ref{sec:databaseandproductivityassessment}, especially Table~\ref{tab:freqoverview}).
Furthermore, regardless of whether one favours a rule-based or a similarity\slash exemplar-based view (see Section~\ref{sec:conditionsonlinkageselectioningerman}), in many cases the use of a PL is highly unlikely for reasons of class memebership or phonotactics.
A clear case are words ending in full vowels, which have an \textit{-s} plural but resist \textit{-s} linkages strongly.
For example, while \textit{Auto-s+kollektion} `car collection' cannot be excluded with absolute certainty, \textit{Auto-kollektion} would be preferred by a huge margin.
Such factors stand in the way of establishing a clearer link between PLs and pluarlity.

Second, we must ask why the corpus-based results were weaker than the split-100 results.
We propose that this is due to the fact that we made a considerable effort to ensure that the stimuli in the experiment consisted of compounds which were most likely novel to the subjects (see Section~\ref{sec:designchoiceofstimuliandparticipants}).
In creating the samples for corpus study, however, we did not differentiate between compounds strongly established in language use and novel compounds.
Therefore, the corpus sample contains lexicalised compounds with conventionalised idiosyncratic linkages.
Of course, we should expect that even these follow the tendency to associate PLs with plural meaning, and thus the added complexities of lexicalised compounds should not be handled by simply filtering compounds which have a high token frequency from concordances.
More theoretcial, empirical, and even methodological work on this is clearly required.


Third, we must ask why PLs differ with respect to their tendency to be used to mark plurality.




Fourth, an explanation of why only the internal plural effect was obversed is needed.
% As \parencite[365--366]{Olsen2015} puts it, ``compounds do not overtly express the grammatical or conceptual relations that exist between their constituents and are, consequently, inherently ambiguous'', but that ``compounds with plural and phrasal first constituents are not uncommon''.


\subsection{Why the differences in L2 Dutch?}

% Why not spoken/PERCEPTION: We only looked at prodcution-side with our sources of data.
% Also, SchreuerEa2004 give reasons why with auditory stimuli no plural connection for PL.
% In general: PL not required for proper interpretation, esp. with internal plural effect.

% This is, in our view, distinct from the question of whether non-head constituents in compounds can establish referents of their own, which \textcite[366]{Olsen2015} denies but \textcite{Dressler1987} demonstrates is possible.

\subsection{Where to go next? Closing comments.}

% More refined corpus studies taking into account the degree to which a compound is novel or established.

% Why not in spoken language (SchreuderEa2004)? This needs further work and experiments from different paradigms.
