\section{Conclusion}
\label{sec:conclusion}

Both in the corpus study (Section~\ref{sec:corpusstudy}) and the split-100 experiment (Section~\ref{sec:split100experiment}), we found evidence that writers prefer to use a PL when N1 necessarily has a plural interpretation due to compound-internal semantics.
We found no evidence for a similar influence of a plural on the whole compound (external plural effect), which could have been semantically motivated in some cases or be a simple compound-internal number agreement tendency.
Researchers have been suspecting this (see Section~\ref{sec:pluralinterpretationofpluraliclinkagesingermananddutch}), and in light of the previous research on Dutch LEs, this does not come as a surprise.
Regardless of their diachronic development, writers use pluralic linking elements as cues to the interpretation of N1+N2 compounds (see also \citealt[212]{BangaEa2013b}).
Because the form of an N1 with a PL is one which writers associate strongly with semantic plurals, they do so also when that form occurs in a compound.
A detailed look at our results reveals more about the actual mechanisms at work.

First, the effect strengths in the corpus study were weak to intermediate, and while the observed effect was rather clear in the split-100 experiment, we are nowhere near a categorical split.
There is also substantial per-speaker variation and uncertainty.
Readers should revisit Figure~\ref{fig:conditions} and Figure~\ref{fig:participants} to convince themselves that this is the case.
So, we have to answer the question of why the overall effect is not stronger.
As pointed out in Section~\ref{sec:pluralinterpretationofpluraliclinkagesingermananddutch}, \textcite{LibbenEa2002} \hl{[Correct reference? Check!]} have found that there is a general tendency to avoid LEs, which also accounts for the high percentage of zero linkages (see Section~\ref{sec:databaseandproductivityassessment}, especially Table~\ref{tab:freqoverview}).
Furthermore, regardless of whether one favours a rule-based or a similarity\slash exemplar-based view (see Section~\ref{sec:conditionsonlinkageselectioningerman}), in many cases the use of a PL is highly unlikely for reasons of class membership or phonotactics.
A clear case are words ending in full vowels, which have an \textit{-s} plural but resist \textit{-s} linkages strongly.
For example, while \textit{Auto-s+kollektion} `car collection' cannot be excluded with absolute certainty, \textit{Auto-kollektion} would be preferred by a huge margin.
Such factors stand in the way of establishing a stronger link between PLs and pluarlity.

Second, we must ask why the corpus-based results were weaker than the split-100 results.
We propose that this is due to the fact that we made a considerable effort to ensure that the stimuli in the experiment consisted of compounds which were most likely novel to the subjects (see Section~\ref{sec:designchoiceofstimuliandparticipants}).
In creating the samples for corpus study, however, we did not differentiate between compounds strongly established in language use and novel compounds.
Therefore, the corpus sample contains lexicalised compounds with conventionalised idiosyncratic linkages.
Of course, we should expect that even lexicalised compounds follow the tendency to associate PLs with plural meaning at least to some extent.
Thus, the added complexities of lexicalised compounds cannot be handled by simply filtering compounds which have a high token frequency from concordances.
More theoretical, empirical, and even methodological work on this is clearly required.

Third, we must ask why PLs differ with respect to their tendency to be used to mark plurality in the corpus study.%
\footnote{The number of possible stimuli in the experiment was too low due to constraints imposed by the experimental procedure to allow a more differentiated look at variation between PLs.}
The order of preference established in Section~\ref{sec:corpusstudy} (especially Figure~\ref{fig:corpuseffects}) was \textit{=e} \textgreater \textit{=er} \textgreater \textit{-er} \textgreater \textit{=} \textgreater \textit{-e} \textgreater \textit{-en} \textgreater \textit{-n}.
The strength of the internal plural effect is thus by and large negatively correlated with the type frequency of nouns which take the respective plural (see Section~\ref{sec:databaseandproductivityassessment}, especially Table~\ref{tab:freqoverview}, and Section~\ref{sec:queries}, especially Figure~\ref{fig:corpusselection}).
Additionally, inflected forms with \textit{-en} and \textit{-n} occur in many positions in German nominal inflection and therefore provide the lowest cue validity for plurality.
For example, there is a dedicated dative marker \textit{-(e)n} attaching to all plurals except \textit{-(e)s}, where it is excluded for phonotactic reasons.
Also, the weak nouns mark all non-nominative singular forms with \textit{-(e)n}.
Finally, the so-called \textit{weak} and \textit{mixed} adjectives end in \textit{-en} in all oblique (dative and genitive) singular forms and the masculine accusative singular.
Contrarily, \textit{-er} is much more exclusively a plural marker (with some exceptions in the so-called \textit{strong} adjectival paradigm), and umlaut is clearly reserved to mark plural in German nominal inflection.%
\footnote{Some comparatives take an umlaut, but the formation of comparatives is clearly something very different from case and number inflection.
Also, case and number inflection comes additionally to the right of comparative markers.}
Finally, stem umlaut is a highly salient cue, whereas \textit{-e} (schwa) and \textit{-(e)n} (usually realised as a syllabic nasal) are the least salient of plural markers.
Taken together, this means that plural interpretation of PLs is stronger with what \textcite{Koepcke1993} calls a higher \textit{Signalstärke} `signal strength'.
His criteria for high signal strength are are high salience, high cue validity, low type frequency, and a high degree  iconicity.%
\footnote{We do not discuss the more complicated notion of iconicity here for space reasons.
By type frequency, Köpcke means the number of different forms in a paradigm, and what we have described as cue validity above encompasses Köpcke's cue validity and type frequency.
For us, type frequency is rather the number of different nouns with which a plural marker occurs.}
We conclude, that writers associate plural meaning more strongly with PLs if the PL itself is more strongly associated with plurality, \ie has greater signal strength.
Furthermore, we consider it likely that the correlation between type frequency in our sense and the increased tendency for plural interpretation of PLs is merely accidental.
The more salient and cue-valid plural patterns happen to be the less type-frequent ones for historical reason.

Fourth, an explanation of why only the internal plural effect was observed is necessary.
In Section~\ref{sec:linkingelementsinprobabilisticmorphology}, we argued that restrictive frameworks tend to elevate universal tendencies to hard universal constraints.
Among the proposed morphological constraints were those requiring that there be no inflection inside products of word formation.
We would like to point out that the tendencies that we have uncovered concern the internally licensed inflectional category of number only (see Section~\ref{sec:pluralinterpretationofpluraliclinkagesingermananddutch}).
The major externally licensed nominal inflectional category, namely case, would still be highly unlikely to occur between N1 and N2, obeying the putative universal constraints.
We suggest that this follows from the nature of compounds.
Case could conceivably only be used to specify the grammatical relation between a compound's constituents, and this is something not usually found in compounds, but rather in complex noun phrases.
Number, on the other hand, has nothing to do with the grammatical relation between the constituents.
Writers use the optional plural marking when the individual and joint conceptual representation of the constituents allows this, but it has nothing to do with syntagmatic grammatical relations.
However, the external plural effect would require that a formed compound be somehow transparent for grammatical categories attaching to its whole.
This is atypical of compounds.

Fifth, we would like to propose an explanation of why our findings seemingly contradict those of \textcite{KoesterEa2004}.
They show in an experiment using event-related potentials (ERP) that mismatches between the semantic (non-)plurality of N1 and PLs\slash NPLs do not lead to the expected N400 effect, concluding that PLs and NPLs are not used by hearers to decode the conceptual structure of N1+N2 compounds.
In contrast to the ERP experiment, we only considered the production side of the effect by looking at usage data from a corpus and preferences in a production-oriented decision task.
Since plurality is only marked optionally within compounds, as we have argued extensively above, hearers cannot and need not rely on PLs as plural markers.
However, the fact that hearers do not rely on plural marking not necessarily has to keep writers from using plural forms where a plural meaning is intended.
This even begs the question of whether the plural is really \textit{marked} or just redundantly indicated by a PL.
Only further research can shed light on such questions.%
\footnote{Some simple experiments could provide clarification about the reception of PLs.
These could involve subjects associating objects or illustrations with truly ambiguous compounds such as \textit{Apfel+teller} `apple plate' and \textit{Äpfel=+teller}.
For more implicit testing, the task might be set up in the visual world paradigm.}

Sixth and finally, \textcite{BangaEa2013b} found that Greman L2 learners of Dutch do not react as strongly to the Dutch PL \textit{-en} as Dutch L1 speakers.
This can be explained nicely given that we have shown that and why Germans attribute the lowest plural interpretability to \textit{-en} in German.
If there really is an interference effect, then the Dutch marker is accidentally identical to the German marker with the lowest signal strength and thus the lowest affinity to a plural interpretation in compounds.
For Dutch speakers, however, given the complete absence of nominal case inflection and much reduced system of plural markers, differences in signal strength are irrelevant.

Our study has helped in showing that German PLs can indeed have a plural interpretation.
More importantly, we have shown that there are differences in when this interpretation is available.
We have also provided cognitively oriented explanations for these differences.
We see many possible routes for future research on the subject.
For example, more refined corpus studies taking into account the degree to which a compound is novel or established should be made.
Also, the differences between the production and the reception side of plurality in compounds need further research, especially considering that for each side, only one study has been conducted.
While there probably is a solid connections between PLs and plurality, this connection's exact nature still has to be characterised fully.
