\section{Function and interpretation}
\label{sec:functionandinterpretation}


As \parencite[365--366]{Olsen2015} puts it, ``compounds do not overtly express the grammatical or conceptual relations that exist between their constituents and are, consequently, inherently ambiguous'', but that ``compounds with plural and phrasal first constituents are not uncommon''.

This is, in our view, distinct from the question of whether non-head constituents in compounds can establish referents of their own, which \textcite[366]{Olsen2015} denies but \textcite{Dressler1987} demonstrates is possible.

A rather extremist position is taken by \textcite{NeefBorgwaldt2012} and \textcite{Neef2015}.
In these papers, the authors dismiss polyfunctional (and probably multifactorial) explanations because, as they say, no stringent system has been proposed which explains exactly under which conditions which linkage is chosen (\egg \citealt[27--29]{NeefBorgwaldt2012}).
Except for a small number of ``exceptions'', the authors seem to suggest that function and form have to stand in a one-to-one relationship.
We have argued in Section~\ref{sec:probabilsticgrammar} that such a view is becoming more and more implausible, and our results clearly support a probabilistic approach to the plural-indicating function of linking elements.


