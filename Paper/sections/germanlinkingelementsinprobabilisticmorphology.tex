\section{Linking elements in probabilistic morphology}
\label{sec:linkingelementsinprobabilisticmorphology}

In this paper, we deal with so-called \textit{linking elements} in German nominal compounds.
Linking elements are inserted between the first and second nominal constituents of a compound, such as \textit{-er} in \textit{Liedertext} `lyrics', literally `song text'.
Linking elements are not obligatory, and many compounds are acceptable with and without such an element.
While many speakers might have individual preferences, \textit{Liedtext} is equally accepted by many speakers.
The distribution of the relatively large number of linking elements has been described in terms of constraints (some of them soft, others virtually categorical) from different angles (\egg \citealt{Fuhrhop1996,Wegener2003,Schluecker2012,NueblingSzczepaniak2013,FuhrhopKuerschner2015}).
Conceptually complete models of linking element selection have been proposed in rule-based systems (possibly with a similarity-based component; see \citealt{DresslerEa2001}) and entirely exemplar-based models \parencite{KrottEa2007}.
See also Section~\ref{sec:linkingelementsingerman}.

It is often assumed that linking elements neither have a grammatical function nor a semantic interpretation.
However, except for some very rare linking elements, a default zero linkage, and a linking element \textit{-(e)s}, all other linking elements are formally identical to the first constituent's plural.
Additionally, a larger number of first constituents alternate between such a pluralic linkage and another non-pluralic linkage (typically zero or \textit{-(e)s}).
This begs the question of whether speaker-hearers or writer-readers (or both) associate a plural meaning with such pluralic linking elements.
So far, researchers have been skeptical or even dismissive about this possibility, and in a perception experiment, \textcite{KoesterEa2004} could not find cues that a plural interpretation of linking elements is triggered in spoken language.
However, for (written) Dutch linking elements, such effects have been demonstrated repeatedly \parencite{SchreuderEa1998,BangaEa2012,BangaEa2013a,BangaEa2013b}.
We present systematic research using corpus data and experimental results showing that German linking elements are indeed used as cues for a plural meaning of the first constituent in written language.

Such findings have, in our view, an impact on morphological theory in general for two highly related reasons.
First, phenomena like the systematic occurrence of inflected forms within products of word-formation such as compounds define the flexibility that any morphological framework or theory needs to offer.
Especially classical generative frameworks (\egg \citealt{Siegel1979,Mohanan1986,Anderson1992,Pinker1999}) tended to implement strong universal tendencies as hard constraints into the architecture of the framework.
As \textcite[391]{Haspelmath2010} puts it, ``most empirical universals are tendencies''.
An example is Siegel's strict layering hypothesis which states that inflection applies derivationally after word formation and that consequently markers of inflectional categories are strictly positioned at the edges of words (more on this in Section~\ref{sec:consequencesofplurallinkagesformorphologicaltheories}).
Such approaches have been criticised for many decades.
\textcite{Pollard1996} astutely criticises the absurdity of needlessly restrictive generative frameworks in syntax, and \textcite{Haspelmath2010} goes even further in denying frameworks their right to exist, demonstrating with many examples how frameworks sacrifice explanatory adequacy for hard-wired allegedly universal resitrictions.
With respect to inflection inside German compounds, we provide a framework-free but differentiated image to be summarised in Section~\ref{sec:functionandinterpretation}.

Second, findings such as ours lend support to a probabilistic view of morphology and grammar in general.
The pile of evidence for the inherent gradedness of grammar has been growing for decades, and \textcite{HayBaayen2005} already summarised an impressive number of studies \hl{on this}, and in syntax, \textcite{Bresnan2007} and subsequent work has radically changed the way empirical syntactic research is conducted.
\textcite[105]{ArndtlappeEa2016} include the selection of linking elements in their list of ``semi-systematic and gradient properties'' of compounds which have been actively researched in recent years and stress the importance empirical work has been playing in this strain of research.
The present study constributes to this \hl{body of work} by showing that there is a -- by now means categorical -- tendency for linkages which take the form of a plural to be interpreted as plurals.
\textcite[107]{ArndtlappeEa2016} address exactly this issue, when they state that ``although there is not and probably never has been a one-to-one correspondence between the form and meaning of compounds, the form does provide a wide variety of information to which humans have access in reaching an interpretation''.
As we will argue in Section~\ref{sec:functionandinterpretation}, it would even be surprising if writer-readers did not pick up on the possibility to use plural forms to denote plural where it makes perfect sence, at least in the absence of strong inhibitory factors.

The paper is structured as follows.
In Section~\ref{sec:linkingelementsingerman}, we review the form and distribution of linking elements in German, and we review positions on their potential plural interpretation.
It includes a discussion of a long tradition of research on Dutch linking elements and their plural interpretation.
The section closes with an overview of our own studies which focus on the preference for pluralic linkages to be used in compounds where the internal conceptual structure of the compound enforces a plural interpretation of the first constituent.
Section~\ref{sec:data} describes a database created by us based on the DECOW16A web corpus.
The database quantifies the frequencies with which a large number of first constituents occurs with pluralic and non-pluralic linkages, and it quantifies each of these noun's productivity with these linkages.
In Section~\ref{sec:corpusstudy}, we use the database to select a large number of alternating first elements and show using a manually annotated corpus sample of approximately $10,000$ compounds containing these first elements that pluralic linkages are indeed cues for semantic plurality.
In Section~\ref{sec:split100experiment}, we report an experiment in the split-100 paradigm which corroborates the findings from the corpus study.
Finally, in Section~\ref{sec:functionandinterpretation}, we discuss the findings in the larger context of probabilistic morphology.

