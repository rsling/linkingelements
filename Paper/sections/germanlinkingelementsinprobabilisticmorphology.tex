\section{German linking elements in probabilistic morphology}
\label{sec:germanlinkingelementsinprobabilisticmorphology}

\paragraph{Short intro to phenomenon}

In this paper, we deal with so-called \textit{linking elements} in German nominal compounds.
Linking elements are inserted between the first and second nominal constituents of a compound, such as \textit{-er} in \textit{Liedertext} `lyrics', literally `song text'.
Linking elements are not obligatory, and many compounds are acceptable with and without such an element.
While many speakers might have individual preferences, \textit{Liedtext} is equally accepted by many speakers.
The distribution of the relatively large number of linking elements has not been modelled in all detail, but many constraints (some of them soft, others virtually categorical) have been described (\egg \citealt{Fuhrhop1996,Wegener2003,Schluecker2012,NueblingSzczepaniak2013,FuhrhopKuerschner2015}); see Section~\ref{sec:linkingelementsingerman}.

It is often assumed that linking elements neither have a grammatical function nor a semantic interpretation.
However, besides some very rare linking elements, a default zero linkage, and an element \textit{-(e)s}, all other linking elements are formally identical to the first constituent's plural.
This begs the question whether speaker-hearers and writer-readers associate a plural meaning with such pluralic linking elements.
So far, researchers have been skeptical or even dismissive about this possibility, although for Dutch, such effects have been demonstrated repeatedly \parencite{SchreuderEa1998,BangaEa2012,BangaEa2013a,BangaEa2013b}.
We present systematic research using corpus data and experimental results showing that German linking elements are indeed used as cues for a plural meaning of the first constituent.

\paragraph{Morphological theories and frameworks (incl. Haspelmath)}

Ah-haaaa!

\paragraph{Probabilism}

\paragraph{Overview of the paper}

The paper is structured as follows.
In Section~\ref{sec:linkingelementsingerman}, we review the form and distribution of linking elements in German, and we review positions on their potential plural interpretation.
It includes a discussion of a long tradition of research on Dutch linking elements and their plural interpretation.
The section closes with an overview of our own studies which focus on the preference for pluralic linkages to be used in compounds where the internal conceptual structure of the compound enforces a plural interpretation of the first constituent.
Section~\ref{sec:data} describes a database created by us based on the DECOW16A web corpus.
The database quantifies the frequencies with which a large number of first constituents occurs with pluralic and non-pluralic linkages, and it quantifies each of these noun's productivity with these linkages.
In Section~\ref{sec:corpusstudy}, we use the database to select $50$ alternating first elements and show using a large manually annotated corpus  sample of compounds containing these that pluralic linkages are indeed cues for semantic plurality.
In Section~\ref{sec:split100experiment}, we report an experiment in the split-100 paradigm which corroborates the findings from the corpus study.
Finally, in Section~\ref{sec:functionandinterpretation}, we discuss the findings in the larger context of probabilistic morphology.
